
\nonstopmode
\newcommand{\keyw}[1]{\ \mbox{\bf #1}\ }
\newcommand{\pname}[1]{\ \mbox{\sc #1}\ }
\newcommand{\vname}[1]{\mbox{\em #1}}
\newcommand{\num}[1]{\ \mbox{#1}\ }
\newcommand{\txt}[1]{\ \mbox{#1}\ }
\newcommand{\ind} {\hspace{5mm}}

\newcounter{algstep}
\newcommand{\firststep}{\setcounter{algstep}{1} \thealgstep.}
\newcommand{\nextstep}{\stepcounter{algstep} \thealgstep.}

\documentclass[12pt]{article}
\usepackage{fullpage}
\usepackage{alltt}
\usepackage{times}
\usepackage{courier}
\usepackage{helvet}
\usepackage{graphicx}
% \usepackage{amsmath}
  

\pagestyle{empty}
\setlength{\parskip}{3mm}

\begin{document}
% \title{AI Haiku}
% \author{Scott Hurlow}
% \date{September, 22 2014}
% \maketitle



% \section{Haiku Algorithm Pseudocode}
% The MakeHaiku algorithm checks if given input text matches haiku
% syllable rules. If it does, it constructs and returns the haiku as a string. We assume
% english grammer rules apply, but the algorithm can be streatched
% to work for other languages and grammers.

  \[
  \begin{array}{lll}
  \multicolumn{2}{l}{\pname{Algorithm\_MakeHaiku}( S )} \\
  & \textnormal{// Input: Array, S, contains a list of strings (each one word)} \\
  & \textnormal{// representing the sentence to be tested.} \\
  % \firststep & \keyw{set} haikuStr = "" \\
  \nextstep & \keyw{set} sylTotal = 0 \\
  \nextstep & \keyw{set} line = 1 \\
  \nextstep & \keyw{set} endIndex = 0 \\ \\
  \nextstep & \keyw{for} i = 0 \keyw{to} S.length - 1: \\
  \nextstep & \ind \keyw{set} word = S[i] \\
  \nextstep & \ind \keyw{set} syl = word.countSyllables() \\ \\
  \nextstep & \ind \keyw{if} line == 1: \\
  \nextstep & \ind \ind \keyw{if} sylTotal == 5: \\
  % \nextstep & \ind \ind \ind lineIndex[1] = i \\
  \nextstep & \ind \ind \ind line++ \\
  \nextstep & \ind \ind \keyw{else if} sylTotal > 5: \textnormal{// if syllable total is greater than 5 but still at line 1}\\
  \nextstep & \ind \ind \ind \keyw{break} \\
  \nextstep & \ind \keyw{else if} line == 2: \\
  \nextstep & \ind \ind \keyw{if} sylTotal == 12: \\
  % \nextstep & \ind \ind \ind lineIndex[2] = i \\
  \nextstep & \ind \ind \ind line++ \\
  \nextstep & \ind \ind \keyw{else if} sylTotal > 12:\\
  \nextstep & \ind \ind \ind \keyw{break} \\
  \nextstep & \ind \keyw{else:} \\
  \nextstep & \ind \ind \keyw{if} sylTotal == 17: \\
  \nextstep & \ind \ind \ind endIndex = i \\
  \nextstep & \ind \ind \ind \keyw{break} \\
  \nextstep & \ind \ind \keyw{else if} sylTotal > 17: \\
  \nextstep & \ind \ind \ind \keyw{break} \\ \\
  \nextstep & \keyw{if} sylTotal == 17: \textnormal{// Vaid haiku! build string and return} \\
  \nextstep & \ind \keyw{set} haikuStr = S[ 0...endIndex ].join() \\
  % \nextstep & \ind \keyw{for} i = 0 \keyw{to} lineIndex[3]:\\
  % \nextstep & \ind \ind haikuStr = haikuStr + S[i] \\
  \nextstep & \ind \keyw{return} haikuStr \\
  \nextstep & \keyw{else:} \textnormal{  //If loop ends with syllable total not equal to 17, S is NOT valid haiku} \\
  \nextstep & \ind \keyw{return} "Error"

  \end{array}
  \]

\end{document}